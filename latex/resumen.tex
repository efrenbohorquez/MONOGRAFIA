\section*{RESUMEN}
\addcontentsline{toc}{section}{RESUMEN}

Los Objetivos de Desarrollo Sostenible (ODS) establecidos por las Naciones Unidas en la Agenda 2030 representan un marco global para abordar los desafíos más apremiantes de la humanidad. En el contexto colombiano, la administración pública de Bogotá D.C. ha asumido el compromiso de implementar estrategias y políticas públicas orientadas al cumplimiento de estos objetivos. 

Esta monografía analiza la implementación y gestión de los ODS en la administración pública del Distrito Capital, examinando las políticas, programas y mecanismos institucionales adoptados entre [PERÍODO DE ESTUDIO]. El estudio se fundamenta en una metodología [TIPO DE METODOLOGÍA] que incluye [DESCRIPCIÓN BREVE DE MÉTODOS].

Los principales hallazgos revelan que [RESULTADOS PRINCIPALES ENCONTRADOS]. Se identificaron fortalezas en [ASPECTOS POSITIVOS] y desafíos en [ASPECTOS A MEJORAR]. Las conclusiones sugieren que [CONCLUSIONES PRINCIPALES] y se recomienda [RECOMENDACIONES PRINCIPALES].

\textbf{Palabras clave:} Objetivos de Desarrollo Sostenible, Administración Pública, Bogotá, Políticas Públicas, Agenda 2030, Desarrollo Sostenible.

\section*{ABSTRACT}
\addcontentsline{toc}{section}{ABSTRACT}

The Sustainable Development Goals (SDGs) established by the United Nations in the 2030 Agenda represent a global framework to address humanity's most pressing challenges. In the Colombian context, the public administration of Bogotá D.C. has assumed the commitment to implement strategies and public policies aimed at achieving these objectives.

This monograph analyzes the implementation and management of the SDGs in the public administration of the Capital District, examining the policies, programs, and institutional mechanisms adopted between [STUDY PERIOD]. The study is based on a [TYPE OF METHODOLOGY] methodology that includes [BRIEF DESCRIPTION OF METHODS].

The main findings reveal that [MAIN RESULTS FOUND]. Strengths were identified in [POSITIVE ASPECTS] and challenges in [ASPECTS TO IMPROVE]. The conclusions suggest that [MAIN CONCLUSIONS] and it is recommended [MAIN RECOMMENDATIONS].

\textbf{Keywords:} Sustainable Development Goals, Public Administration, Bogotá, Public Policy, 2030 Agenda, Sustainable Development.
