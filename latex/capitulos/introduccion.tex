\section{INTRODUCCIÓN}

\subsection{Planteamiento del Problema}

Los Objetivos de Desarrollo Sostenible (ODS), adoptados por las Naciones Unidas en septiembre de 2015 como parte de la Agenda 2030, constituyen un marco global de acción para abordar los desafíos más urgentes de la humanidad en materia de desarrollo sostenible. Estos 17 objetivos interconectados buscan erradicar la pobreza, proteger el planeta y garantizar la prosperidad para todos.

En el contexto colombiano, y particularmente en Bogotá D.C., la administración pública enfrenta el desafío de traducir estos objetivos globales en políticas públicas efectivas y medibles. La capital colombiana, como centro político y económico del país, desempeña un papel fundamental en la implementación de estrategias que contribuyan al cumplimiento de la Agenda 2030.

Sin embargo, la implementación de los ODS en el ámbito de la administración pública presenta múltiples desafíos: la articulación entre diferentes niveles de gobierno, la asignación de recursos, la medición del impacto, la participación ciudadana y la coordinación intersectorial. Además, la complejidad de los problemas urbanos de Bogotá requiere enfoques innovadores y sostenibles que alineen las prioridades locales con los compromisos globales.

\textbf{Pregunta de investigación:} ¿Cómo ha sido la implementación y gestión de los Objetivos de Desarrollo Sostenible en la administración pública de Bogotá D.C. y cuáles son los principales logros, desafíos y oportunidades de mejora en el período [DEFINIR PERÍODO]?

\subsection{Justificación}

Esta investigación se justifica por varias razones fundamentales:

\textbf{Relevancia académica:} Existe una brecha en la literatura académica sobre la implementación específica de los ODS en el contexto de la administración pública local colombiana, particularmente en Bogotá.

\textbf{Importancia práctica:} Los resultados de esta investigación pueden proporcionar insumos valiosos para mejorar las políticas públicas y los mecanismos de implementación de los ODS en el Distrito Capital.

\textbf{Pertinencia temporal:} En el contexto actual de la Agenda 2030, es crucial evaluar el progreso y identificar las mejores prácticas para acelerar el cumplimiento de los objetivos.

\textbf{Impacto social:} Una mejor implementación de los ODS puede contribuir significativamente al bienestar de los más de 8 millones de habitantes de Bogotá y su área metropolitana.

\subsection{Objetivos}

\subsubsection{Objetivo General}

Analizar la implementación y gestión de los Objetivos de Desarrollo Sostenible en la administración pública de Bogotá D.C., identificando logros, desafíos y oportunidades de mejora en el marco de la Agenda 2030.

\subsubsection{Objetivos Específicos}

\begin{enumerate}
    \item Caracterizar el marco institucional y normativo que sustenta la implementación de los ODS en la administración pública de Bogotá.
    
    \item Identificar y analizar las principales políticas públicas, programas y proyectos implementados por la administración distrital en relación con los ODS.
    
    \item Evaluar los mecanismos de coordinación intersectorial e intergubernamental establecidos para la gestión de los ODS.
    
    \item Analizar los sistemas de monitoreo, seguimiento y evaluación utilizados para medir el progreso en el cumplimiento de los ODS.
    
    \item Identificar los principales logros, desafíos y barreras en la implementación de los ODS en el contexto bogotano.
    
    \item Formular recomendaciones para fortalecer la gestión pública orientada al cumplimiento de los ODS en Bogotá.
\end{enumerate}

\subsection{Alcance y Limitaciones}

\subsubsection{Alcance}

Esta investigación se enfoca específicamente en:

\begin{itemize}
    \item La administración pública del Distrito Capital de Bogotá
    \item El período comprendido entre [DEFINIR PERÍODO DE ANÁLISIS]
    \item Los 17 Objetivos de Desarrollo Sostenible de la Agenda 2030
    \item Las políticas públicas, programas y proyectos distritales relacionados con los ODS
    \item Los mecanismos institucionales de coordinación y seguimiento
\end{itemize}

\subsubsection{Limitaciones}

Las principales limitaciones del estudio incluyen:

\begin{itemize}
    \item Disponibilidad y acceso a información oficial actualizada
    \item Tiempo limitado para realizar el análisis completo
    \item Restricciones presupuestarias para el trabajo de campo
    \item Posible sesgo en las fuentes de información institucional
    \item Cambios en las administraciones públicas durante el período de estudio
\end{itemize}

\subsection{Estructura de la Monografía}

Este documento se organiza en seis capítulos principales:

\textbf{Capítulo 1 - Introducción:} Presenta el planteamiento del problema, justificación, objetivos y metodología.

\textbf{Capítulo 2 - Marco Teórico:} Desarrolla los fundamentos conceptuales sobre ODS, administración pública y políticas públicas.

\textbf{Capítulo 3 - Metodología:} Describe el diseño metodológico, técnicas de investigación e instrumentos utilizados.

\textbf{Capítulo 4 - Desarrollo:} Analiza el marco institucional y las políticas públicas implementadas en Bogotá.

\textbf{Capítulo 5 - Resultados:} Presenta los hallazgos de la investigación y su análisis.

\textbf{Capítulo 6 - Conclusiones:} Sintetiza los principales resultados y formula recomendaciones.
