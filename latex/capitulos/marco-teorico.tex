\section{MARCO TEÓRICO}

\subsection{Antecedentes}

\subsubsection{Los Objetivos de Desarrollo del Milenio (ODM)}

Los Objetivos de Desarrollo Sostenible tienen sus raíces en los Objetivos de Desarrollo del Milenio (ODM), establecidos en el año 2000. Los ODM constituyeron el primer marco global de desarrollo que logró un consenso internacional significativo, estableciendo ocho objetivos específicos para ser alcanzados en 2015 \citep{naciones_unidas_2000}.

La experiencia de implementación de los ODM proporcionó lecciones valiosas que fueron incorporadas en el diseño de los ODS, incluyendo la necesidad de mayor participación de los gobiernos locales, la importancia de la medición y el seguimiento, y la relevancia de adaptar los objetivos globales a los contextos nacionales y subnacionales.

\subsubsection{Antecedentes de Investigación}

\textbf{Estudios Internacionales:}

\citet{autor_ejemplo_2020} analizaron la implementación de los ODS en ciudades europeas, encontrando que los factores clave de éxito incluyen el liderazgo político, la participación ciudadana y la coordinación intersectorial.

\citet{otro_autor_2019} examinaron los mecanismos de localización de los ODS en América Latina, destacando la importancia de adaptar los indicadores globales a las realidades locales.

\textbf{Estudios Nacionales:}

En Colombia, \citet{investigador_colombiano_2021} evaluó el marco institucional nacional para la implementación de los ODS, identificando fortalezas en la coordinación interinstitucional pero debilidades en los sistemas de monitoreo.

\citet{grupo_investigacion_2020} analizó la incorporación de los ODS en los planes de desarrollo territorial, encontrando heterogeneidad en los enfoques adoptados por diferentes entidades territoriales.

\subsection{Bases Teóricas}

\subsubsection{La Agenda 2030 y los Objetivos de Desarrollo Sostenible}

Los Objetivos de Desarrollo Sostenible constituyen un marco integral que aborda los desafíos globales más apremiantes. Adoptados por 193 países en septiembre de 2015, los 17 ODS y sus 169 metas buscan ``no dejar a nadie atrás'' en el camino hacia el desarrollo sostenible \citep{naciones_unidas_2015}.

Los ODS se fundamentan en tres dimensiones interconectadas del desarrollo sostenible:

\begin{itemize}
    \item \textbf{Dimensión económica:} Promoción del crecimiento económico sostenible, el empleo productivo y la innovación.
    \item \textbf{Dimensión social:} Erradicación de la pobreza, reducción de las desigualdades y promoción del bienestar.
    \item \textbf{Dimensión ambiental:} Protección del medio ambiente y gestión sostenible de los recursos naturales.
\end{itemize}

\subsubsection{Teoría de la Administración Pública}

La administración pública, como disciplina y práctica, se encarga de la implementación de las políticas públicas y la gestión de los recursos del Estado para satisfacer las necesidades de la sociedad \citep{autor_administracion_2018}.

En el contexto de los ODS, la administración pública juega un papel fundamental como:

\begin{enumerate}
    \item \textbf{Articulador:} Coordinando acciones entre diferentes sectores y niveles de gobierno.
    \item \textbf{Implementador:} Ejecutando políticas y programas orientados al cumplimiento de los objetivos.
    \item \textbf{Monitor:} Estableciendo sistemas de seguimiento y evaluación del progreso.
    \item \textbf{Facilitador:} Promoviendo la participación de actores no gubernamentales.
\end{enumerate}

\subsubsection{Teoría de Políticas Públicas}

Las políticas públicas son ``cursos de acción desarrollados por cuerpos gubernamentales y sus funcionarios'' \citep{anderson_2015}. En el marco de los ODS, las políticas públicas constituyen el instrumento principal para traducir los compromisos globales en acciones concretas a nivel local.

El ciclo de las políticas públicas incluye las siguientes etapas:

\begin{enumerate}
    \item \textbf{Identificación del problema:} Reconocimiento de situaciones que requieren intervención pública.
    \item \textbf{Formulación:} Diseño de alternativas de solución y selección de la más adecuada.
    \item \textbf{Implementación:} Puesta en marcha de la política seleccionada.
    \item \textbf{Evaluación:} Valoración de los resultados e impactos obtenidos.
\end{enumerate}

\subsubsection{Gobernanza y Desarrollo Sostenible}

La gobernanza se refiere a ``las tradiciones e instituciones mediante las cuales se ejerce la autoridad en un país'' \citep{banco_mundial_2017}. Una gobernanza efectiva es fundamental para la implementación exitosa de los ODS.

Los principios de buena gobernanza incluyen:

\begin{itemize}
    \item Transparencia y rendición de cuentas
    \item Participación ciudadana
    \item Estado de derecho
    \item Efectividad y eficiencia
    \item Equidad e inclusión
\end{itemize}

\subsection{Marco Conceptual}

\subsubsection{Definiciones Clave}

\textbf{Objetivos de Desarrollo Sostenible (ODS):} Marco global de 17 objetivos interconectados diseñados para ser un ``plan maestro para conseguir un futuro sostenible para todos'' \citep{naciones_unidas_2015}.

\textbf{Localización de los ODS:} Proceso de definir, implementar y monitorear estrategias a nivel local para alcanzar la Agenda 2030, desde el establecimiento de objetivos locales hasta la determinación de los medios de implementación \citep{cglu_2018}.

\textbf{Administración Pública:} Sistema de organizaciones públicas que implementan las políticas públicas y proporcionan servicios públicos a los ciudadanos.

\textbf{Políticas Públicas:} Conjunto de acciones implementadas en el marco de planes y programas gubernamentales, dirigidas a satisfacer demandas y necesidades de la sociedad.

\textbf{Desarrollo Sostenible:} Desarrollo que satisface las necesidades del presente sin comprometer la capacidad de las futuras generaciones para satisfacer sus propias necesidades \citep{brundtland_1987}.

\subsection{Estado del Arte}

\subsubsection{Tendencias Globales en la Implementación de ODS}

La implementación de los ODS a nivel global ha mostrado avances dispares. El Informe de los Objetivos de Desarrollo Sostenible 2023 \citep{naciones_unidas_2023} indica que el mundo está lejos de alcanzar los ODS para 2030, con solo el 15\% de las metas en camino de cumplirse.

Las principales tendencias identificadas incluyen:

\begin{itemize}
    \item Mayor atención a la localización de los ODS
    \item Desarrollo de indicadores complementarios a nivel local
    \item Uso de tecnologías digitales para el monitoreo
    \item Enfoque en la participación multi-actor
    \item Integración de los ODS en la planificación urbana
\end{itemize}

\subsubsection{Experiencias Latinoamericanas}

En América Latina, diversos países han desarrollado estrategias nacionales para la implementación de los ODS. Experiencias destacadas incluyen:

\textbf{México:} Desarrollo de un Sistema de Información de los ODS que permite el seguimiento desagregado a nivel estatal y municipal.

\textbf{Chile:} Creación de una agenda nacional de desarrollo sostenible con metas adaptadas al contexto nacional.

\textbf{Costa Rica:} Implementación de un modelo de articulación territorial que vincula los ODS con la planificación local.

\subsubsection{El Caso Colombiano}

Colombia ha sido reconocida internacionalmente por sus esfuerzos en la implementación de los ODS. El país estableció la Comisión Interinstitucional de Alto Nivel para el alistamiento y la efectiva implementación de la Agenda 2030 y sus ODS (CIAT-ODS) \citep{gobierno_colombia_2018}.

Los principales avances incluyen:

\begin{itemize}
    \item Alineación de los ODS con el Plan Nacional de Desarrollo
    \item Desarrollo de indicadores nacionales para el seguimiento
    \item Territorialización de los ODS en entidades territoriales
    \item Participación activa en foros internacionales sobre ODS
\end{itemize}
